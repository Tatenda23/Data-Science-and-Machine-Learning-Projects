\documentclass[11pt]{article}

    \usepackage[breakable]{tcolorbox}
    \usepackage{parskip} % Stop auto-indenting (to mimic markdown behaviour)
    
    \usepackage{iftex}
    \ifPDFTeX
    	\usepackage[T1]{fontenc}
    	\usepackage{mathpazo}
    \else
    	\usepackage{fontspec}
    \fi

    % Basic figure setup, for now with no caption control since it's done
    % automatically by Pandoc (which extracts ![](path) syntax from Markdown).
    \usepackage{graphicx}
    % Maintain compatibility with old templates. Remove in nbconvert 6.0
    \let\Oldincludegraphics\includegraphics
    % Ensure that by default, figures have no caption (until we provide a
    % proper Figure object with a Caption API and a way to capture that
    % in the conversion process - todo).
    \usepackage{caption}
    \DeclareCaptionFormat{nocaption}{}
    \captionsetup{format=nocaption,aboveskip=0pt,belowskip=0pt}

    \usepackage[Export]{adjustbox} % Used to constrain images to a maximum size
    \adjustboxset{max size={0.9\linewidth}{0.9\paperheight}}
    \usepackage{float}
    \floatplacement{figure}{H} % forces figures to be placed at the correct location
    \usepackage{xcolor} % Allow colors to be defined
    \usepackage{enumerate} % Needed for markdown enumerations to work
    \usepackage{geometry} % Used to adjust the document margins
    \usepackage{amsmath} % Equations
    \usepackage{amssymb} % Equations
    \usepackage{textcomp} % defines textquotesingle
    % Hack from http://tex.stackexchange.com/a/47451/13684:
    \AtBeginDocument{%
        \def\PYZsq{\textquotesingle}% Upright quotes in Pygmentized code
    }
    \usepackage{upquote} % Upright quotes for verbatim code
    \usepackage{eurosym} % defines \euro
    \usepackage[mathletters]{ucs} % Extended unicode (utf-8) support
    \usepackage{fancyvrb} % verbatim replacement that allows latex
    \usepackage{grffile} % extends the file name processing of package graphics 
                         % to support a larger range
    \makeatletter % fix for grffile with XeLaTeX
    \def\Gread@@xetex#1{%
      \IfFileExists{"\Gin@base".bb}%
      {\Gread@eps{\Gin@base.bb}}%
      {\Gread@@xetex@aux#1}%
    }
    \makeatother

    % The hyperref package gives us a pdf with properly built
    % internal navigation ('pdf bookmarks' for the table of contents,
    % internal cross-reference links, web links for URLs, etc.)
    \usepackage{hyperref}
    % The default LaTeX title has an obnoxious amount of whitespace. By default,
    % titling removes some of it. It also provides customization options.
    \usepackage{titling}
    \usepackage{longtable} % longtable support required by pandoc >1.10
    \usepackage{booktabs}  % table support for pandoc > 1.12.2
    \usepackage[inline]{enumitem} % IRkernel/repr support (it uses the enumerate* environment)
    \usepackage[normalem]{ulem} % ulem is needed to support strikethroughs (\sout)
                                % normalem makes italics be italics, not underlines
    \usepackage{mathrsfs}
    

    
    % Colors for the hyperref package
    \definecolor{urlcolor}{rgb}{0,.145,.698}
    \definecolor{linkcolor}{rgb}{.71,0.21,0.01}
    \definecolor{citecolor}{rgb}{.12,.54,.11}

    % ANSI colors
    \definecolor{ansi-black}{HTML}{3E424D}
    \definecolor{ansi-black-intense}{HTML}{282C36}
    \definecolor{ansi-red}{HTML}{E75C58}
    \definecolor{ansi-red-intense}{HTML}{B22B31}
    \definecolor{ansi-green}{HTML}{00A250}
    \definecolor{ansi-green-intense}{HTML}{007427}
    \definecolor{ansi-yellow}{HTML}{DDB62B}
    \definecolor{ansi-yellow-intense}{HTML}{B27D12}
    \definecolor{ansi-blue}{HTML}{208FFB}
    \definecolor{ansi-blue-intense}{HTML}{0065CA}
    \definecolor{ansi-magenta}{HTML}{D160C4}
    \definecolor{ansi-magenta-intense}{HTML}{A03196}
    \definecolor{ansi-cyan}{HTML}{60C6C8}
    \definecolor{ansi-cyan-intense}{HTML}{258F8F}
    \definecolor{ansi-white}{HTML}{C5C1B4}
    \definecolor{ansi-white-intense}{HTML}{A1A6B2}
    \definecolor{ansi-default-inverse-fg}{HTML}{FFFFFF}
    \definecolor{ansi-default-inverse-bg}{HTML}{000000}

    % commands and environments needed by pandoc snippets
    % extracted from the output of `pandoc -s`
    \providecommand{\tightlist}{%
      \setlength{\itemsep}{0pt}\setlength{\parskip}{0pt}}
    \DefineVerbatimEnvironment{Highlighting}{Verbatim}{commandchars=\\\{\}}
    % Add ',fontsize=\small' for more characters per line
    \newenvironment{Shaded}{}{}
    \newcommand{\KeywordTok}[1]{\textcolor[rgb]{0.00,0.44,0.13}{\textbf{{#1}}}}
    \newcommand{\DataTypeTok}[1]{\textcolor[rgb]{0.56,0.13,0.00}{{#1}}}
    \newcommand{\DecValTok}[1]{\textcolor[rgb]{0.25,0.63,0.44}{{#1}}}
    \newcommand{\BaseNTok}[1]{\textcolor[rgb]{0.25,0.63,0.44}{{#1}}}
    \newcommand{\FloatTok}[1]{\textcolor[rgb]{0.25,0.63,0.44}{{#1}}}
    \newcommand{\CharTok}[1]{\textcolor[rgb]{0.25,0.44,0.63}{{#1}}}
    \newcommand{\StringTok}[1]{\textcolor[rgb]{0.25,0.44,0.63}{{#1}}}
    \newcommand{\CommentTok}[1]{\textcolor[rgb]{0.38,0.63,0.69}{\textit{{#1}}}}
    \newcommand{\OtherTok}[1]{\textcolor[rgb]{0.00,0.44,0.13}{{#1}}}
    \newcommand{\AlertTok}[1]{\textcolor[rgb]{1.00,0.00,0.00}{\textbf{{#1}}}}
    \newcommand{\FunctionTok}[1]{\textcolor[rgb]{0.02,0.16,0.49}{{#1}}}
    \newcommand{\RegionMarkerTok}[1]{{#1}}
    \newcommand{\ErrorTok}[1]{\textcolor[rgb]{1.00,0.00,0.00}{\textbf{{#1}}}}
    \newcommand{\NormalTok}[1]{{#1}}
    
    % Additional commands for more recent versions of Pandoc
    \newcommand{\ConstantTok}[1]{\textcolor[rgb]{0.53,0.00,0.00}{{#1}}}
    \newcommand{\SpecialCharTok}[1]{\textcolor[rgb]{0.25,0.44,0.63}{{#1}}}
    \newcommand{\VerbatimStringTok}[1]{\textcolor[rgb]{0.25,0.44,0.63}{{#1}}}
    \newcommand{\SpecialStringTok}[1]{\textcolor[rgb]{0.73,0.40,0.53}{{#1}}}
    \newcommand{\ImportTok}[1]{{#1}}
    \newcommand{\DocumentationTok}[1]{\textcolor[rgb]{0.73,0.13,0.13}{\textit{{#1}}}}
    \newcommand{\AnnotationTok}[1]{\textcolor[rgb]{0.38,0.63,0.69}{\textbf{\textit{{#1}}}}}
    \newcommand{\CommentVarTok}[1]{\textcolor[rgb]{0.38,0.63,0.69}{\textbf{\textit{{#1}}}}}
    \newcommand{\VariableTok}[1]{\textcolor[rgb]{0.10,0.09,0.49}{{#1}}}
    \newcommand{\ControlFlowTok}[1]{\textcolor[rgb]{0.00,0.44,0.13}{\textbf{{#1}}}}
    \newcommand{\OperatorTok}[1]{\textcolor[rgb]{0.40,0.40,0.40}{{#1}}}
    \newcommand{\BuiltInTok}[1]{{#1}}
    \newcommand{\ExtensionTok}[1]{{#1}}
    \newcommand{\PreprocessorTok}[1]{\textcolor[rgb]{0.74,0.48,0.00}{{#1}}}
    \newcommand{\AttributeTok}[1]{\textcolor[rgb]{0.49,0.56,0.16}{{#1}}}
    \newcommand{\InformationTok}[1]{\textcolor[rgb]{0.38,0.63,0.69}{\textbf{\textit{{#1}}}}}
    \newcommand{\WarningTok}[1]{\textcolor[rgb]{0.38,0.63,0.69}{\textbf{\textit{{#1}}}}}
    
    
    % Define a nice break command that doesn't care if a line doesn't already
    % exist.
    \def\br{\hspace*{\fill} \\* }
    % Math Jax compatibility definitions
    \def\gt{>}
    \def\lt{<}
    \let\Oldtex\TeX
    \let\Oldlatex\LaTeX
    \renewcommand{\TeX}{\textrm{\Oldtex}}
    \renewcommand{\LaTeX}{\textrm{\Oldlatex}}
    % Document parameters
    % Document title
    \title{Customer Segmentation in the US}
    
    
    
    
    
% Pygments definitions
\makeatletter
\def\PY@reset{\let\PY@it=\relax \let\PY@bf=\relax%
    \let\PY@ul=\relax \let\PY@tc=\relax%
    \let\PY@bc=\relax \let\PY@ff=\relax}
\def\PY@tok#1{\csname PY@tok@#1\endcsname}
\def\PY@toks#1+{\ifx\relax#1\empty\else%
    \PY@tok{#1}\expandafter\PY@toks\fi}
\def\PY@do#1{\PY@bc{\PY@tc{\PY@ul{%
    \PY@it{\PY@bf{\PY@ff{#1}}}}}}}
\def\PY#1#2{\PY@reset\PY@toks#1+\relax+\PY@do{#2}}

\expandafter\def\csname PY@tok@w\endcsname{\def\PY@tc##1{\textcolor[rgb]{0.73,0.73,0.73}{##1}}}
\expandafter\def\csname PY@tok@c\endcsname{\let\PY@it=\textit\def\PY@tc##1{\textcolor[rgb]{0.25,0.50,0.50}{##1}}}
\expandafter\def\csname PY@tok@cp\endcsname{\def\PY@tc##1{\textcolor[rgb]{0.74,0.48,0.00}{##1}}}
\expandafter\def\csname PY@tok@k\endcsname{\let\PY@bf=\textbf\def\PY@tc##1{\textcolor[rgb]{0.00,0.50,0.00}{##1}}}
\expandafter\def\csname PY@tok@kp\endcsname{\def\PY@tc##1{\textcolor[rgb]{0.00,0.50,0.00}{##1}}}
\expandafter\def\csname PY@tok@kt\endcsname{\def\PY@tc##1{\textcolor[rgb]{0.69,0.00,0.25}{##1}}}
\expandafter\def\csname PY@tok@o\endcsname{\def\PY@tc##1{\textcolor[rgb]{0.40,0.40,0.40}{##1}}}
\expandafter\def\csname PY@tok@ow\endcsname{\let\PY@bf=\textbf\def\PY@tc##1{\textcolor[rgb]{0.67,0.13,1.00}{##1}}}
\expandafter\def\csname PY@tok@nb\endcsname{\def\PY@tc##1{\textcolor[rgb]{0.00,0.50,0.00}{##1}}}
\expandafter\def\csname PY@tok@nf\endcsname{\def\PY@tc##1{\textcolor[rgb]{0.00,0.00,1.00}{##1}}}
\expandafter\def\csname PY@tok@nc\endcsname{\let\PY@bf=\textbf\def\PY@tc##1{\textcolor[rgb]{0.00,0.00,1.00}{##1}}}
\expandafter\def\csname PY@tok@nn\endcsname{\let\PY@bf=\textbf\def\PY@tc##1{\textcolor[rgb]{0.00,0.00,1.00}{##1}}}
\expandafter\def\csname PY@tok@ne\endcsname{\let\PY@bf=\textbf\def\PY@tc##1{\textcolor[rgb]{0.82,0.25,0.23}{##1}}}
\expandafter\def\csname PY@tok@nv\endcsname{\def\PY@tc##1{\textcolor[rgb]{0.10,0.09,0.49}{##1}}}
\expandafter\def\csname PY@tok@no\endcsname{\def\PY@tc##1{\textcolor[rgb]{0.53,0.00,0.00}{##1}}}
\expandafter\def\csname PY@tok@nl\endcsname{\def\PY@tc##1{\textcolor[rgb]{0.63,0.63,0.00}{##1}}}
\expandafter\def\csname PY@tok@ni\endcsname{\let\PY@bf=\textbf\def\PY@tc##1{\textcolor[rgb]{0.60,0.60,0.60}{##1}}}
\expandafter\def\csname PY@tok@na\endcsname{\def\PY@tc##1{\textcolor[rgb]{0.49,0.56,0.16}{##1}}}
\expandafter\def\csname PY@tok@nt\endcsname{\let\PY@bf=\textbf\def\PY@tc##1{\textcolor[rgb]{0.00,0.50,0.00}{##1}}}
\expandafter\def\csname PY@tok@nd\endcsname{\def\PY@tc##1{\textcolor[rgb]{0.67,0.13,1.00}{##1}}}
\expandafter\def\csname PY@tok@s\endcsname{\def\PY@tc##1{\textcolor[rgb]{0.73,0.13,0.13}{##1}}}
\expandafter\def\csname PY@tok@sd\endcsname{\let\PY@it=\textit\def\PY@tc##1{\textcolor[rgb]{0.73,0.13,0.13}{##1}}}
\expandafter\def\csname PY@tok@si\endcsname{\let\PY@bf=\textbf\def\PY@tc##1{\textcolor[rgb]{0.73,0.40,0.53}{##1}}}
\expandafter\def\csname PY@tok@se\endcsname{\let\PY@bf=\textbf\def\PY@tc##1{\textcolor[rgb]{0.73,0.40,0.13}{##1}}}
\expandafter\def\csname PY@tok@sr\endcsname{\def\PY@tc##1{\textcolor[rgb]{0.73,0.40,0.53}{##1}}}
\expandafter\def\csname PY@tok@ss\endcsname{\def\PY@tc##1{\textcolor[rgb]{0.10,0.09,0.49}{##1}}}
\expandafter\def\csname PY@tok@sx\endcsname{\def\PY@tc##1{\textcolor[rgb]{0.00,0.50,0.00}{##1}}}
\expandafter\def\csname PY@tok@m\endcsname{\def\PY@tc##1{\textcolor[rgb]{0.40,0.40,0.40}{##1}}}
\expandafter\def\csname PY@tok@gh\endcsname{\let\PY@bf=\textbf\def\PY@tc##1{\textcolor[rgb]{0.00,0.00,0.50}{##1}}}
\expandafter\def\csname PY@tok@gu\endcsname{\let\PY@bf=\textbf\def\PY@tc##1{\textcolor[rgb]{0.50,0.00,0.50}{##1}}}
\expandafter\def\csname PY@tok@gd\endcsname{\def\PY@tc##1{\textcolor[rgb]{0.63,0.00,0.00}{##1}}}
\expandafter\def\csname PY@tok@gi\endcsname{\def\PY@tc##1{\textcolor[rgb]{0.00,0.63,0.00}{##1}}}
\expandafter\def\csname PY@tok@gr\endcsname{\def\PY@tc##1{\textcolor[rgb]{1.00,0.00,0.00}{##1}}}
\expandafter\def\csname PY@tok@ge\endcsname{\let\PY@it=\textit}
\expandafter\def\csname PY@tok@gs\endcsname{\let\PY@bf=\textbf}
\expandafter\def\csname PY@tok@gp\endcsname{\let\PY@bf=\textbf\def\PY@tc##1{\textcolor[rgb]{0.00,0.00,0.50}{##1}}}
\expandafter\def\csname PY@tok@go\endcsname{\def\PY@tc##1{\textcolor[rgb]{0.53,0.53,0.53}{##1}}}
\expandafter\def\csname PY@tok@gt\endcsname{\def\PY@tc##1{\textcolor[rgb]{0.00,0.27,0.87}{##1}}}
\expandafter\def\csname PY@tok@err\endcsname{\def\PY@bc##1{\setlength{\fboxsep}{0pt}\fcolorbox[rgb]{1.00,0.00,0.00}{1,1,1}{\strut ##1}}}
\expandafter\def\csname PY@tok@kc\endcsname{\let\PY@bf=\textbf\def\PY@tc##1{\textcolor[rgb]{0.00,0.50,0.00}{##1}}}
\expandafter\def\csname PY@tok@kd\endcsname{\let\PY@bf=\textbf\def\PY@tc##1{\textcolor[rgb]{0.00,0.50,0.00}{##1}}}
\expandafter\def\csname PY@tok@kn\endcsname{\let\PY@bf=\textbf\def\PY@tc##1{\textcolor[rgb]{0.00,0.50,0.00}{##1}}}
\expandafter\def\csname PY@tok@kr\endcsname{\let\PY@bf=\textbf\def\PY@tc##1{\textcolor[rgb]{0.00,0.50,0.00}{##1}}}
\expandafter\def\csname PY@tok@bp\endcsname{\def\PY@tc##1{\textcolor[rgb]{0.00,0.50,0.00}{##1}}}
\expandafter\def\csname PY@tok@fm\endcsname{\def\PY@tc##1{\textcolor[rgb]{0.00,0.00,1.00}{##1}}}
\expandafter\def\csname PY@tok@vc\endcsname{\def\PY@tc##1{\textcolor[rgb]{0.10,0.09,0.49}{##1}}}
\expandafter\def\csname PY@tok@vg\endcsname{\def\PY@tc##1{\textcolor[rgb]{0.10,0.09,0.49}{##1}}}
\expandafter\def\csname PY@tok@vi\endcsname{\def\PY@tc##1{\textcolor[rgb]{0.10,0.09,0.49}{##1}}}
\expandafter\def\csname PY@tok@vm\endcsname{\def\PY@tc##1{\textcolor[rgb]{0.10,0.09,0.49}{##1}}}
\expandafter\def\csname PY@tok@sa\endcsname{\def\PY@tc##1{\textcolor[rgb]{0.73,0.13,0.13}{##1}}}
\expandafter\def\csname PY@tok@sb\endcsname{\def\PY@tc##1{\textcolor[rgb]{0.73,0.13,0.13}{##1}}}
\expandafter\def\csname PY@tok@sc\endcsname{\def\PY@tc##1{\textcolor[rgb]{0.73,0.13,0.13}{##1}}}
\expandafter\def\csname PY@tok@dl\endcsname{\def\PY@tc##1{\textcolor[rgb]{0.73,0.13,0.13}{##1}}}
\expandafter\def\csname PY@tok@s2\endcsname{\def\PY@tc##1{\textcolor[rgb]{0.73,0.13,0.13}{##1}}}
\expandafter\def\csname PY@tok@sh\endcsname{\def\PY@tc##1{\textcolor[rgb]{0.73,0.13,0.13}{##1}}}
\expandafter\def\csname PY@tok@s1\endcsname{\def\PY@tc##1{\textcolor[rgb]{0.73,0.13,0.13}{##1}}}
\expandafter\def\csname PY@tok@mb\endcsname{\def\PY@tc##1{\textcolor[rgb]{0.40,0.40,0.40}{##1}}}
\expandafter\def\csname PY@tok@mf\endcsname{\def\PY@tc##1{\textcolor[rgb]{0.40,0.40,0.40}{##1}}}
\expandafter\def\csname PY@tok@mh\endcsname{\def\PY@tc##1{\textcolor[rgb]{0.40,0.40,0.40}{##1}}}
\expandafter\def\csname PY@tok@mi\endcsname{\def\PY@tc##1{\textcolor[rgb]{0.40,0.40,0.40}{##1}}}
\expandafter\def\csname PY@tok@il\endcsname{\def\PY@tc##1{\textcolor[rgb]{0.40,0.40,0.40}{##1}}}
\expandafter\def\csname PY@tok@mo\endcsname{\def\PY@tc##1{\textcolor[rgb]{0.40,0.40,0.40}{##1}}}
\expandafter\def\csname PY@tok@ch\endcsname{\let\PY@it=\textit\def\PY@tc##1{\textcolor[rgb]{0.25,0.50,0.50}{##1}}}
\expandafter\def\csname PY@tok@cm\endcsname{\let\PY@it=\textit\def\PY@tc##1{\textcolor[rgb]{0.25,0.50,0.50}{##1}}}
\expandafter\def\csname PY@tok@cpf\endcsname{\let\PY@it=\textit\def\PY@tc##1{\textcolor[rgb]{0.25,0.50,0.50}{##1}}}
\expandafter\def\csname PY@tok@c1\endcsname{\let\PY@it=\textit\def\PY@tc##1{\textcolor[rgb]{0.25,0.50,0.50}{##1}}}
\expandafter\def\csname PY@tok@cs\endcsname{\let\PY@it=\textit\def\PY@tc##1{\textcolor[rgb]{0.25,0.50,0.50}{##1}}}

\def\PYZbs{\char`\\}
\def\PYZus{\char`\_}
\def\PYZob{\char`\{}
\def\PYZcb{\char`\}}
\def\PYZca{\char`\^}
\def\PYZam{\char`\&}
\def\PYZlt{\char`\<}
\def\PYZgt{\char`\>}
\def\PYZsh{\char`\#}
\def\PYZpc{\char`\%}
\def\PYZdl{\char`\$}
\def\PYZhy{\char`\-}
\def\PYZsq{\char`\'}
\def\PYZdq{\char`\"}
\def\PYZti{\char`\~}
% for compatibility with earlier versions
\def\PYZat{@}
\def\PYZlb{[}
\def\PYZrb{]}
\makeatother


    % For linebreaks inside Verbatim environment from package fancyvrb. 
    \makeatletter
        \newbox\Wrappedcontinuationbox 
        \newbox\Wrappedvisiblespacebox 
        \newcommand*\Wrappedvisiblespace {\textcolor{red}{\textvisiblespace}} 
        \newcommand*\Wrappedcontinuationsymbol {\textcolor{red}{\llap{\tiny$\m@th\hookrightarrow$}}} 
        \newcommand*\Wrappedcontinuationindent {3ex } 
        \newcommand*\Wrappedafterbreak {\kern\Wrappedcontinuationindent\copy\Wrappedcontinuationbox} 
        % Take advantage of the already applied Pygments mark-up to insert 
        % potential linebreaks for TeX processing. 
        %        {, <, #, %, $, ' and ": go to next line. 
        %        _, }, ^, &, >, - and ~: stay at end of broken line. 
        % Use of \textquotesingle for straight quote. 
        \newcommand*\Wrappedbreaksatspecials {% 
            \def\PYGZus{\discretionary{\char`\_}{\Wrappedafterbreak}{\char`\_}}% 
            \def\PYGZob{\discretionary{}{\Wrappedafterbreak\char`\{}{\char`\{}}% 
            \def\PYGZcb{\discretionary{\char`\}}{\Wrappedafterbreak}{\char`\}}}% 
            \def\PYGZca{\discretionary{\char`\^}{\Wrappedafterbreak}{\char`\^}}% 
            \def\PYGZam{\discretionary{\char`\&}{\Wrappedafterbreak}{\char`\&}}% 
            \def\PYGZlt{\discretionary{}{\Wrappedafterbreak\char`\<}{\char`\<}}% 
            \def\PYGZgt{\discretionary{\char`\>}{\Wrappedafterbreak}{\char`\>}}% 
            \def\PYGZsh{\discretionary{}{\Wrappedafterbreak\char`\#}{\char`\#}}% 
            \def\PYGZpc{\discretionary{}{\Wrappedafterbreak\char`\%}{\char`\%}}% 
            \def\PYGZdl{\discretionary{}{\Wrappedafterbreak\char`\$}{\char`\$}}% 
            \def\PYGZhy{\discretionary{\char`\-}{\Wrappedafterbreak}{\char`\-}}% 
            \def\PYGZsq{\discretionary{}{\Wrappedafterbreak\textquotesingle}{\textquotesingle}}% 
            \def\PYGZdq{\discretionary{}{\Wrappedafterbreak\char`\"}{\char`\"}}% 
            \def\PYGZti{\discretionary{\char`\~}{\Wrappedafterbreak}{\char`\~}}% 
        } 
        % Some characters . , ; ? ! / are not pygmentized. 
        % This macro makes them "active" and they will insert potential linebreaks 
        \newcommand*\Wrappedbreaksatpunct {% 
            \lccode`\~`\.\lowercase{\def~}{\discretionary{\hbox{\char`\.}}{\Wrappedafterbreak}{\hbox{\char`\.}}}% 
            \lccode`\~`\,\lowercase{\def~}{\discretionary{\hbox{\char`\,}}{\Wrappedafterbreak}{\hbox{\char`\,}}}% 
            \lccode`\~`\;\lowercase{\def~}{\discretionary{\hbox{\char`\;}}{\Wrappedafterbreak}{\hbox{\char`\;}}}% 
            \lccode`\~`\:\lowercase{\def~}{\discretionary{\hbox{\char`\:}}{\Wrappedafterbreak}{\hbox{\char`\:}}}% 
            \lccode`\~`\?\lowercase{\def~}{\discretionary{\hbox{\char`\?}}{\Wrappedafterbreak}{\hbox{\char`\?}}}% 
            \lccode`\~`\!\lowercase{\def~}{\discretionary{\hbox{\char`\!}}{\Wrappedafterbreak}{\hbox{\char`\!}}}% 
            \lccode`\~`\/\lowercase{\def~}{\discretionary{\hbox{\char`\/}}{\Wrappedafterbreak}{\hbox{\char`\/}}}% 
            \catcode`\.\active
            \catcode`\,\active 
            \catcode`\;\active
            \catcode`\:\active
            \catcode`\?\active
            \catcode`\!\active
            \catcode`\/\active 
            \lccode`\~`\~ 	
        }
    \makeatother

    \let\OriginalVerbatim=\Verbatim
    \makeatletter
    \renewcommand{\Verbatim}[1][1]{%
        %\parskip\z@skip
        \sbox\Wrappedcontinuationbox {\Wrappedcontinuationsymbol}%
        \sbox\Wrappedvisiblespacebox {\FV@SetupFont\Wrappedvisiblespace}%
        \def\FancyVerbFormatLine ##1{\hsize\linewidth
            \vtop{\raggedright\hyphenpenalty\z@\exhyphenpenalty\z@
                \doublehyphendemerits\z@\finalhyphendemerits\z@
                \strut ##1\strut}%
        }%
        % If the linebreak is at a space, the latter will be displayed as visible
        % space at end of first line, and a continuation symbol starts next line.
        % Stretch/shrink are however usually zero for typewriter font.
        \def\FV@Space {%
            \nobreak\hskip\z@ plus\fontdimen3\font minus\fontdimen4\font
            \discretionary{\copy\Wrappedvisiblespacebox}{\Wrappedafterbreak}
            {\kern\fontdimen2\font}%
        }%
        
        % Allow breaks at special characters using \PYG... macros.
        \Wrappedbreaksatspecials
        % Breaks at punctuation characters . , ; ? ! and / need catcode=\active 	
        \OriginalVerbatim[#1,codes*=\Wrappedbreaksatpunct]%
    }
    \makeatother

    % Exact colors from NB
    \definecolor{incolor}{HTML}{303F9F}
    \definecolor{outcolor}{HTML}{D84315}
    \definecolor{cellborder}{HTML}{CFCFCF}
    \definecolor{cellbackground}{HTML}{F7F7F7}
    
    % prompt
    \makeatletter
    \newcommand{\boxspacing}{\kern\kvtcb@left@rule\kern\kvtcb@boxsep}
    \makeatother
    \newcommand{\prompt}[4]{
        \ttfamily\llap{{\color{#2}[#3]:\hspace{3pt}#4}}\vspace{-\baselineskip}
    }
    

    
    % Prevent overflowing lines due to hard-to-break entities
    \sloppy 
    % Setup hyperref package
    \hypersetup{
      breaklinks=true,  % so long urls are correctly broken across lines
      colorlinks=true,
      urlcolor=urlcolor,
      linkcolor=linkcolor,
      citecolor=citecolor,
      }
    % Slightly bigger margins than the latex defaults
    
    \geometry{verbose,tmargin=1in,bmargin=1in,lmargin=1in,rmargin=1in}
    
    

\begin{document}
    
    \maketitle
    
    

    
    \hypertarget{introduction}{%
\section{Introduction}\label{introduction}}

    In this project, i am going to build a K-Means model to create clusters
of respondents to the Survey of Consumer Finances. I will examine all
the features, selecting five to create clusters with. After building the
model and choosing an appropriate number of clusters, i will create
visualizations for multi-dimensional clusters in a 2D scatter plot using
principal component analysis (PCA).

    \begin{tcolorbox}[breakable, size=fbox, boxrule=1pt, pad at break*=1mm,colback=cellbackground, colframe=cellborder]
\prompt{In}{incolor}{1}{\boxspacing}
\begin{Verbatim}[commandchars=\\\{\}]
\PY{c+c1}{\PYZsh{} Import libraries here}
\PY{k+kn}{import} \PY{n+nn}{pandas} \PY{k}{as} \PY{n+nn}{pd}
\PY{k+kn}{import} \PY{n+nn}{plotly}\PY{n+nn}{.}\PY{n+nn}{express} \PY{k}{as} \PY{n+nn}{px}
\PY{k+kn}{from} \PY{n+nn}{scipy}\PY{n+nn}{.}\PY{n+nn}{stats}\PY{n+nn}{.}\PY{n+nn}{mstats} \PY{k+kn}{import} \PY{n}{trimmed\PYZus{}var}
\PY{k+kn}{from} \PY{n+nn}{sklearn}\PY{n+nn}{.}\PY{n+nn}{cluster} \PY{k+kn}{import} \PY{n}{KMeans}
\PY{k+kn}{from} \PY{n+nn}{sklearn}\PY{n+nn}{.}\PY{n+nn}{decomposition} \PY{k+kn}{import} \PY{n}{PCA}
\PY{k+kn}{from} \PY{n+nn}{sklearn}\PY{n+nn}{.}\PY{n+nn}{metrics} \PY{k+kn}{import} \PY{n}{silhouette\PYZus{}score}
\PY{k+kn}{from} \PY{n+nn}{sklearn}\PY{n+nn}{.}\PY{n+nn}{pipeline} \PY{k+kn}{import} \PY{n}{make\PYZus{}pipeline}
\PY{k+kn}{from} \PY{n+nn}{sklearn}\PY{n+nn}{.}\PY{n+nn}{preprocessing} \PY{k+kn}{import} \PY{n}{StandardScaler}
\end{Verbatim}
\end{tcolorbox}

    \hypertarget{prepare-data}{%
\section{Prepare Data}\label{prepare-data}}

    First, i need to import the file, scfp2019excel.zip. I will use the
pandas read\_csv () method to import and decompress the data file.

    \begin{tcolorbox}[breakable, size=fbox, boxrule=1pt, pad at break*=1mm,colback=cellbackground, colframe=cellborder]
\prompt{In}{incolor}{2}{\boxspacing}
\begin{Verbatim}[commandchars=\\\{\}]
\PY{n}{df} \PY{o}{=} \PY{n}{pd}\PY{o}{.}\PY{n}{read\PYZus{}csv}\PY{p}{(}\PY{l+s+s2}{\PYZdq{}}\PY{l+s+s2}{/home/tatenda/Desktop/ds\PYZus{}projects/scfp2019excel.zip}\PY{l+s+s2}{\PYZdq{}}\PY{p}{)}
\PY{n+nb}{print}\PY{p}{(}\PY{l+s+s2}{\PYZdq{}}\PY{l+s+s2}{df shape:}\PY{l+s+s2}{\PYZdq{}}\PY{p}{,} \PY{n}{df}\PY{o}{.}\PY{n}{shape}\PY{p}{)}
\PY{n}{df}\PY{o}{.}\PY{n}{head}\PY{p}{(}\PY{p}{)}
\end{Verbatim}
\end{tcolorbox}

    \begin{Verbatim}[commandchars=\\\{\}]
df shape: (28885, 351)
    \end{Verbatim}

            \begin{tcolorbox}[breakable, size=fbox, boxrule=.5pt, pad at break*=1mm, opacityfill=0]
\prompt{Out}{outcolor}{2}{\boxspacing}
\begin{Verbatim}[commandchars=\\\{\}]
   YY1  Y1          WGT  HHSEX  AGE  AGECL  EDUC  EDCL  MARRIED  KIDS  {\ldots}  \textbackslash{}
0    1  11  6119.779308      2   75      6    12     4        2     0  {\ldots}
1    1  12  4712.374912      2   75      6    12     4        2     0  {\ldots}
2    1  13  5145.224455      2   75      6    12     4        2     0  {\ldots}
3    1  14  5297.663412      2   75      6    12     4        2     0  {\ldots}
4    1  15  4761.812371      2   75      6    12     4        2     0  {\ldots}

   NWCAT  INCCAT  ASSETCAT  NINCCAT  NINC2CAT  NWPCTLECAT  INCPCTLECAT  \textbackslash{}
0      5       3         6        3         2          10            6
1      5       3         6        3         1          10            5
2      5       3         6        3         1          10            5
3      5       2         6        2         1          10            4
4      5       3         6        3         1          10            5

   NINCPCTLECAT  INCQRTCAT  NINCQRTCAT
0             6          3           3
1             5          2           2
2             5          2           2
3             4          2           2
4             5          2           2

[5 rows x 351 columns]
\end{Verbatim}
\end{tcolorbox}
        
    Below is a function function that returns a DataFrame of households
whose net worth is less than \$2 million and that have been turned down
for credit or feared being denied credit in the past 5 years (
``TURNFEAR'' and ``NETWORTH'' columns). According to the data
dictionary, TURNFEAR==1 implies households that have been turned down or
feared being turned down for credit.

    \begin{tcolorbox}[breakable, size=fbox, boxrule=1pt, pad at break*=1mm,colback=cellbackground, colframe=cellborder]
\prompt{In}{incolor}{3}{\boxspacing}
\begin{Verbatim}[commandchars=\\\{\}]
\PY{k}{def} \PY{n+nf}{wrangle}\PY{p}{(}\PY{n}{filepath}\PY{p}{)}\PY{p}{:}
    \PY{c+c1}{\PYZsh{}Read file into DataFrame}
    \PY{n}{df} \PY{o}{=} \PY{n}{pd}\PY{o}{.}\PY{n}{read\PYZus{}csv}\PY{p}{(}\PY{n}{filepath}\PY{p}{)}
    \PY{n}{mask} \PY{o}{=} \PY{p}{(}\PY{n}{df}\PY{p}{[}\PY{l+s+s2}{\PYZdq{}}\PY{l+s+s2}{TURNFEAR}\PY{l+s+s2}{\PYZdq{}}\PY{p}{]}\PY{o}{==}\PY{l+m+mi}{1}\PY{p}{)} \PY{o}{\PYZam{}} \PY{p}{(}\PY{n}{df}\PY{p}{[}\PY{l+s+s2}{\PYZdq{}}\PY{l+s+s2}{NETWORTH}\PY{l+s+s2}{\PYZdq{}}\PY{p}{]}\PY{o}{\PYZlt{}}\PY{l+m+mf}{2e6}\PY{p}{)}
    \PY{n}{df} \PY{o}{=} \PY{n}{df}\PY{p}{[}\PY{n}{mask}\PY{p}{]}
    \PY{k}{return} \PY{n}{df}
\end{Verbatim}
\end{tcolorbox}

    \begin{tcolorbox}[breakable, size=fbox, boxrule=1pt, pad at break*=1mm,colback=cellbackground, colframe=cellborder]
\prompt{In}{incolor}{4}{\boxspacing}
\begin{Verbatim}[commandchars=\\\{\}]
\PY{n}{df} \PY{o}{=} \PY{n}{wrangle}\PY{p}{(}\PY{l+s+s2}{\PYZdq{}}\PY{l+s+s2}{/home/tatenda/Desktop/ds\PYZus{}projects/scfp2019excel.zip}\PY{l+s+s2}{\PYZdq{}}\PY{p}{)}
\PY{n+nb}{print}\PY{p}{(}\PY{n}{df}\PY{o}{.}\PY{n}{shape}\PY{p}{)}
\PY{n}{df}\PY{o}{.}\PY{n}{head}\PY{p}{(}\PY{p}{)}
\end{Verbatim}
\end{tcolorbox}

    \begin{Verbatim}[commandchars=\\\{\}]
(4418, 351)
    \end{Verbatim}

            \begin{tcolorbox}[breakable, size=fbox, boxrule=.5pt, pad at break*=1mm, opacityfill=0]
\prompt{Out}{outcolor}{4}{\boxspacing}
\begin{Verbatim}[commandchars=\\\{\}]
   YY1  Y1          WGT  HHSEX  AGE  AGECL  EDUC  EDCL  MARRIED  KIDS  {\ldots}  \textbackslash{}
5    2  21  3790.476607      1   50      3     8     2        1     3  {\ldots}
6    2  22  3798.868505      1   50      3     8     2        1     3  {\ldots}
7    2  23  3799.468393      1   50      3     8     2        1     3  {\ldots}
8    2  24  3788.076005      1   50      3     8     2        1     3  {\ldots}
9    2  25  3793.066589      1   50      3     8     2        1     3  {\ldots}

   NWCAT  INCCAT  ASSETCAT  NINCCAT  NINC2CAT  NWPCTLECAT  INCPCTLECAT  \textbackslash{}
5      1       2         1        2         1           1            4
6      1       2         1        2         1           1            4
7      1       2         1        2         1           1            4
8      1       2         1        2         1           1            4
9      1       2         1        2         1           1            4

   NINCPCTLECAT  INCQRTCAT  NINCQRTCAT
5             4          2           2
6             3          2           2
7             4          2           2
8             4          2           2
9             4          2           2

[5 rows x 351 columns]
\end{Verbatim}
\end{tcolorbox}
        
    \hypertarget{explore}{%
\subsection{Explore}\label{explore}}

    We want to make clusters using several features, but which of the 351
features should we choose? Often times, this decision will be made for
you. For example, a stakeholder could give you a list of the features
that are most important to them. If you don't have that limitation,
though, another way to choose the best features for clustering is to
determine which numerical features have the largest variance. That's
what i will do here.

    Now we calculate the variance for all the features in df, and create a
Series top\_ten\_var with the 10 features with the largest variance.

    \begin{tcolorbox}[breakable, size=fbox, boxrule=1pt, pad at break*=1mm,colback=cellbackground, colframe=cellborder]
\prompt{In}{incolor}{5}{\boxspacing}
\begin{Verbatim}[commandchars=\\\{\}]
\PY{c+c1}{\PYZsh{} Calculate variance, get 10 largest features}
\PY{n}{top\PYZus{}ten\PYZus{}var} \PY{o}{=} \PY{n}{df}\PY{o}{.}\PY{n}{var}\PY{p}{(}\PY{p}{)}\PY{o}{.}\PY{n}{sort\PYZus{}values}\PY{p}{(}\PY{p}{)}\PY{o}{.}\PY{n}{tail}\PY{p}{(}\PY{l+m+mi}{10}\PY{p}{)}
\PY{n}{top\PYZus{}ten\PYZus{}var}
\end{Verbatim}
\end{tcolorbox}

            \begin{tcolorbox}[breakable, size=fbox, boxrule=.5pt, pad at break*=1mm, opacityfill=0]
\prompt{Out}{outcolor}{5}{\boxspacing}
\begin{Verbatim}[commandchars=\\\{\}]
PLOAN1      1.140894e+10
ACTBUS      1.251892e+10
BUS         1.256643e+10
KGTOTAL     1.346475e+10
DEBT        1.848252e+10
NHNFIN      2.254163e+10
HOUSES      2.388459e+10
NETWORTH    4.847029e+10
NFIN        5.713939e+10
ASSET       8.303967e+10
dtype: float64
\end{Verbatim}
\end{tcolorbox}
        
    It is better and easier to see the above information in the form of a
visualisation. So i will use plotly express to create a horizontal bar
chart of top\_ten\_var.

    \begin{tcolorbox}[breakable, size=fbox, boxrule=1pt, pad at break*=1mm,colback=cellbackground, colframe=cellborder]
\prompt{In}{incolor}{6}{\boxspacing}
\begin{Verbatim}[commandchars=\\\{\}]
\PY{c+c1}{\PYZsh{} Create horizontal bar chart of `top\PYZus{}ten\PYZus{}var`}
\PY{n}{fig} \PY{o}{=} \PY{n}{px}\PY{o}{.}\PY{n}{bar}\PY{p}{(}
    \PY{n}{x}\PY{o}{=}\PY{n}{top\PYZus{}ten\PYZus{}var}\PY{p}{,}
    \PY{n}{y}\PY{o}{=}\PY{n}{top\PYZus{}ten\PYZus{}var}\PY{o}{.}\PY{n}{index}\PY{p}{,}
    \PY{n}{title}\PY{o}{=}\PY{l+s+s2}{\PYZdq{}}\PY{l+s+s2}{SCF: High Variance Features}\PY{l+s+s2}{\PYZdq{}}
\PY{p}{)}
\PY{n}{fig}\PY{o}{.}\PY{n}{update\PYZus{}layout}\PY{p}{(}\PY{n}{xaxis\PYZus{}title}\PY{o}{=}\PY{l+s+s2}{\PYZdq{}}\PY{l+s+s2}{Variance}\PY{l+s+s2}{\PYZdq{}}\PY{p}{,} \PY{n}{yaxis\PYZus{}title}\PY{o}{=}\PY{l+s+s2}{\PYZdq{}}\PY{l+s+s2}{Feature}\PY{l+s+s2}{\PYZdq{}}\PY{p}{)}
\PY{n}{fig}\PY{o}{.}\PY{n}{show}\PY{p}{(}\PY{p}{)}
\end{Verbatim}
\end{tcolorbox}

    
    
    
    
    One thing that we are seeing in this project is that many of the wealth
indicators are highly skewed, with a few outlier households having
enormous wealth. Those outliers can affect our measure of variance.
Let's see if that's the case with one of the features from
top\_five\_var

    I will use plotly express to create a horizontal boxplot of ``NHNFIN''
to determine if the values are skewed

    \begin{tcolorbox}[breakable, size=fbox, boxrule=1pt, pad at break*=1mm,colback=cellbackground, colframe=cellborder]
\prompt{In}{incolor}{7}{\boxspacing}
\begin{Verbatim}[commandchars=\\\{\}]
\PY{c+c1}{\PYZsh{} Create a boxplot of `NHNFIN`}
\PY{n}{fig} \PY{o}{=} \PY{n}{px}\PY{o}{.}\PY{n}{box}\PY{p}{(}
    \PY{n}{data\PYZus{}frame}\PY{o}{=}\PY{n}{df}\PY{p}{,}
    \PY{n}{x}\PY{o}{=}\PY{l+s+s2}{\PYZdq{}}\PY{l+s+s2}{NHNFIN}\PY{l+s+s2}{\PYZdq{}}\PY{p}{,}
    \PY{n}{title}\PY{o}{=} \PY{l+s+s2}{\PYZdq{}}\PY{l+s+s2}{Distribution of Non\PYZhy{}home, Non\PYZhy{}Financial Assets}\PY{l+s+s2}{\PYZdq{}}
\PY{p}{)}
\PY{n}{fig}\PY{o}{.}\PY{n}{update\PYZus{}layout}\PY{p}{(}\PY{n}{xaxis\PYZus{}title}\PY{o}{=}\PY{l+s+s2}{\PYZdq{}}\PY{l+s+s2}{Value [\PYZdl{}]}\PY{l+s+s2}{\PYZdq{}}\PY{p}{)}

\PY{n}{fig}\PY{o}{.}\PY{n}{show}\PY{p}{(}\PY{p}{)}
\end{Verbatim}
\end{tcolorbox}

    
    
    The dataset is massively right-skewed because of the huge outliers on
the right side of the distribution. Even though we already excluded
households with a high net worth with our \texttt{wrangle} function, the
variance is still being distorted by some extreme outliers.

The best way to deal with this is to look at the \textbf{trimmed
variance}, where we remove extreme values before calculating variance.
We can do this using the \texttt{trimmed\_variance} function from the
\texttt{SciPy} library.

    Let's calculate the trimmed variance for the features in df. The
calculation excludes the top and bottom 10\% of observations. Then we
will create a Series top\_ten\_trim\_var with the 10 features with the
largest variance.

    \begin{tcolorbox}[breakable, size=fbox, boxrule=1pt, pad at break*=1mm,colback=cellbackground, colframe=cellborder]
\prompt{In}{incolor}{8}{\boxspacing}
\begin{Verbatim}[commandchars=\\\{\}]
\PY{c+c1}{\PYZsh{} Calculate trimmed variance}
\PY{n}{top\PYZus{}ten\PYZus{}trim\PYZus{}var} \PY{o}{=} \PY{n}{df}\PY{o}{.}\PY{n}{apply}\PY{p}{(}\PY{n}{trimmed\PYZus{}var}\PY{p}{)}\PY{o}{.}\PY{n}{sort\PYZus{}values}\PY{p}{(}\PY{p}{)}\PY{o}{.}\PY{n}{tail}\PY{p}{(}\PY{l+m+mi}{10}\PY{p}{)}

\PY{n}{top\PYZus{}ten\PYZus{}trim\PYZus{}var}
\end{Verbatim}
\end{tcolorbox}

            \begin{tcolorbox}[breakable, size=fbox, boxrule=.5pt, pad at break*=1mm, opacityfill=0]
\prompt{Out}{outcolor}{8}{\boxspacing}
\begin{Verbatim}[commandchars=\\\{\}]
WAGEINC     5.550737e+08
HOMEEQ      7.338377e+08
NH\_MORT     1.333125e+09
MRTHEL      1.380468e+09
PLOAN1      1.441968e+09
DEBT        3.089865e+09
NETWORTH    3.099929e+09
HOUSES      4.978660e+09
NFIN        8.456442e+09
ASSET       1.175370e+10
dtype: float64
\end{Verbatim}
\end{tcolorbox}
        
    Let's use plotly express to create a horizontal bar chart of
top\_ten\_trim\_var

    \begin{tcolorbox}[breakable, size=fbox, boxrule=1pt, pad at break*=1mm,colback=cellbackground, colframe=cellborder]
\prompt{In}{incolor}{9}{\boxspacing}
\begin{Verbatim}[commandchars=\\\{\}]
\PY{c+c1}{\PYZsh{} Create horizontal bar chart of `top\PYZus{}ten\PYZus{}trim\PYZus{}var`}
\PY{n}{fig} \PY{o}{=} \PY{n}{px}\PY{o}{.}\PY{n}{bar}\PY{p}{(}
    \PY{n}{x}\PY{o}{=}\PY{n}{top\PYZus{}ten\PYZus{}trim\PYZus{}var}\PY{p}{,}
    \PY{n}{y}\PY{o}{=}\PY{n}{top\PYZus{}ten\PYZus{}trim\PYZus{}var}\PY{o}{.}\PY{n}{index}\PY{p}{,}
    \PY{n}{title}\PY{o}{=}\PY{l+s+s2}{\PYZdq{}}\PY{l+s+s2}{SCF: High Variance Features}\PY{l+s+s2}{\PYZdq{}}
\PY{p}{)}

\PY{n}{fig}\PY{o}{.}\PY{n}{update\PYZus{}layout}\PY{p}{(}\PY{n}{xaxis\PYZus{}title}\PY{o}{=} \PY{l+s+s2}{\PYZdq{}}\PY{l+s+s2}{Trimmed Variance}\PY{l+s+s2}{\PYZdq{}}\PY{p}{,} \PY{n}{yaxis\PYZus{}title} \PY{o}{=}\PY{l+s+s2}{\PYZdq{}}\PY{l+s+s2}{Feature}\PY{l+s+s2}{\PYZdq{}}\PY{p}{)}

\PY{n}{fig}\PY{o}{.}\PY{n}{show}\PY{p}{(}\PY{p}{)}
\end{Verbatim}
\end{tcolorbox}

    
    
    There are three things to notice in this plot. First, the variances have
decreased a lot. In our previous chart, the x-axis went up to
\textbackslash\$80 billion; this one goes up to \textbackslash\$12
billion. Second, the top 10 features have changed a bit. All the
features relating to business ownership (\texttt{"...BUS"}) are gone.
Finally, we can see that there are big differences in variance from
feature to feature. For example, the variance for \texttt{"WAGEINC"} is
around than \textbackslash\$500 million, while the variance for
\texttt{"ASSET"} is nearly \textbackslash\$12 billion. In other words,
these features have completely different scales. This is something that
we'll need to address before we can make good clusters.

    Let's generate a list high\_var\_cols with the column names of the five
features with the highest trimmed variance.

    \begin{tcolorbox}[breakable, size=fbox, boxrule=1pt, pad at break*=1mm,colback=cellbackground, colframe=cellborder]
\prompt{In}{incolor}{10}{\boxspacing}
\begin{Verbatim}[commandchars=\\\{\}]
\PY{n}{high\PYZus{}var\PYZus{}cols} \PY{o}{=} \PY{n}{top\PYZus{}ten\PYZus{}trim\PYZus{}var}\PY{o}{.}\PY{n}{tail}\PY{p}{(}\PY{l+m+mi}{5}\PY{p}{)}\PY{o}{.}\PY{n}{index}\PY{o}{.}\PY{n}{to\PYZus{}list}\PY{p}{(}\PY{p}{)}
\PY{n}{high\PYZus{}var\PYZus{}cols}
\end{Verbatim}
\end{tcolorbox}

            \begin{tcolorbox}[breakable, size=fbox, boxrule=.5pt, pad at break*=1mm, opacityfill=0]
\prompt{Out}{outcolor}{10}{\boxspacing}
\begin{Verbatim}[commandchars=\\\{\}]
['DEBT', 'NETWORTH', 'HOUSES', 'NFIN', 'ASSET']
\end{Verbatim}
\end{tcolorbox}
        
    \hypertarget{split}{%
\subsection{Split}\label{split}}

Let's create the feature matrix X. It should contain the five columns in
high\_var\_cols

    \begin{tcolorbox}[breakable, size=fbox, boxrule=1pt, pad at break*=1mm,colback=cellbackground, colframe=cellborder]
\prompt{In}{incolor}{11}{\boxspacing}
\begin{Verbatim}[commandchars=\\\{\}]
\PY{n}{X} \PY{o}{=} \PY{n}{df}\PY{p}{[}\PY{n}{high\PYZus{}var\PYZus{}cols}\PY{p}{]}
\PY{n+nb}{print}\PY{p}{(}\PY{l+s+s2}{\PYZdq{}}\PY{l+s+s2}{X shape:}\PY{l+s+s2}{\PYZdq{}}\PY{p}{,} \PY{n}{X}\PY{o}{.}\PY{n}{shape}\PY{p}{)}
\PY{n}{X}\PY{o}{.}\PY{n}{head}\PY{p}{(}\PY{p}{)}
\end{Verbatim}
\end{tcolorbox}

    \begin{Verbatim}[commandchars=\\\{\}]
X shape: (4418, 5)
    \end{Verbatim}

            \begin{tcolorbox}[breakable, size=fbox, boxrule=.5pt, pad at break*=1mm, opacityfill=0]
\prompt{Out}{outcolor}{11}{\boxspacing}
\begin{Verbatim}[commandchars=\\\{\}]
      DEBT  NETWORTH  HOUSES     NFIN    ASSET
5  12200.0   -6710.0     0.0   3900.0   5490.0
6  12600.0   -4710.0     0.0   6300.0   7890.0
7  15300.0   -8115.0     0.0   5600.0   7185.0
8  14100.0   -2510.0     0.0  10000.0  11590.0
9  15400.0   -5715.0     0.0   8100.0   9685.0
\end{Verbatim}
\end{tcolorbox}
        
    \hypertarget{build-model}{%
\section{Build Model}\label{build-model}}

    \hypertarget{iterate}{%
\subsection{Iterate}\label{iterate}}

    During the EDA, we saw that we had a scale issue among our features.
That issue can make it harder to cluster the data, so we'll need to fix
that to help our analysis along. One strategy we can use is
\textbf{standardization}, a statistical method for putting all the
variables in a dataset on the same scale. Let's explore how that works
here. Later, we'll incorporate it into our model pipeline.

    Create a StandardScaler transformer, use it to transform the data in X,
and then put the transformed data into a DataFrame named X\_scaled.

    \begin{tcolorbox}[breakable, size=fbox, boxrule=1pt, pad at break*=1mm,colback=cellbackground, colframe=cellborder]
\prompt{In}{incolor}{12}{\boxspacing}
\begin{Verbatim}[commandchars=\\\{\}]
\PY{c+c1}{\PYZsh{} Instantiate transformer}
\PY{n}{ss} \PY{o}{=} \PY{n}{StandardScaler}\PY{p}{(}\PY{p}{)}

\PY{c+c1}{\PYZsh{} Transform `X`}
\PY{n}{X\PYZus{}scaled\PYZus{}data} \PY{o}{=} \PY{n}{ss}\PY{o}{.}\PY{n}{fit\PYZus{}transform}\PY{p}{(}\PY{n}{X}\PY{p}{)}  

\PY{c+c1}{\PYZsh{} Put `X\PYZus{}scaled\PYZus{}data` into DataFrame}
\PY{n}{X\PYZus{}scaled} \PY{o}{=} \PY{n}{pd}\PY{o}{.}\PY{n}{DataFrame}\PY{p}{(}\PY{n}{X\PYZus{}scaled\PYZus{}data}\PY{p}{,}\PY{n}{columns}\PY{o}{=}\PY{n}{X}\PY{o}{.}\PY{n}{columns}\PY{p}{)}

\PY{n+nb}{print}\PY{p}{(}\PY{l+s+s2}{\PYZdq{}}\PY{l+s+s2}{X\PYZus{}scaled shape:}\PY{l+s+s2}{\PYZdq{}}\PY{p}{,} \PY{n}{X\PYZus{}scaled}\PY{o}{.}\PY{n}{shape}\PY{p}{)}
\PY{n}{X\PYZus{}scaled}\PY{o}{.}\PY{n}{head}\PY{p}{(}\PY{p}{)}
\end{Verbatim}
\end{tcolorbox}

    \begin{Verbatim}[commandchars=\\\{\}]
X\_scaled shape: (4418, 5)
    \end{Verbatim}

            \begin{tcolorbox}[breakable, size=fbox, boxrule=.5pt, pad at break*=1mm, opacityfill=0]
\prompt{Out}{outcolor}{12}{\boxspacing}
\begin{Verbatim}[commandchars=\\\{\}]
       DEBT  NETWORTH   HOUSES      NFIN     ASSET
0 -0.445075 -0.377486 -0.48231 -0.474583 -0.498377
1 -0.442132 -0.368401 -0.48231 -0.464541 -0.490047
2 -0.422270 -0.383868 -0.48231 -0.467470 -0.492494
3 -0.431097 -0.358407 -0.48231 -0.449061 -0.477206
4 -0.421534 -0.372966 -0.48231 -0.457010 -0.483818
\end{Verbatim}
\end{tcolorbox}
        
    Let's use a for loop to build and train a K-Means model where
n\_clusters ranges from 2 to 12 (inclusive). The model includes a
StandardScaler. Each time a model is trained, calculate the inertia and
add it to the list inertia\_errors, then calculate the silhouette score
and add it to the list silhouette\_scores.

    \begin{tcolorbox}[breakable, size=fbox, boxrule=1pt, pad at break*=1mm,colback=cellbackground, colframe=cellborder]
\prompt{In}{incolor}{13}{\boxspacing}
\begin{Verbatim}[commandchars=\\\{\}]
\PY{n}{n\PYZus{}clusters} \PY{o}{=} \PY{n+nb}{range}\PY{p}{(}\PY{l+m+mi}{2}\PY{p}{,} \PY{l+m+mi}{12}\PY{p}{)}
\PY{n}{inertia\PYZus{}errors} \PY{o}{=} \PY{p}{[}\PY{p}{]}
\PY{n}{silhouette\PYZus{}scores} \PY{o}{=} \PY{p}{[}\PY{p}{]}

\PY{c+c1}{\PYZsh{} Add `for` loop to train model and calculate inertia, silhouette score.}
\PY{k}{for} \PY{n}{k} \PY{o+ow}{in} \PY{n}{n\PYZus{}clusters}\PY{p}{:}
    \PY{c+c1}{\PYZsh{}Build model}
    \PY{n}{model} \PY{o}{=} \PY{n}{make\PYZus{}pipeline}\PY{p}{(}\PY{n}{StandardScaler}\PY{p}{(}\PY{p}{)}\PY{p}{,} \PY{n}{KMeans}\PY{p}{(}\PY{n}{n\PYZus{}clusters}\PY{o}{=}\PY{n}{k}\PY{p}{,} \PY{n}{random\PYZus{}state}\PY{o}{=}\PY{l+m+mi}{42}\PY{p}{)}\PY{p}{)}
    \PY{c+c1}{\PYZsh{}Train model}
    \PY{n}{model}\PY{o}{.}\PY{n}{fit}\PY{p}{(}\PY{n}{X}\PY{p}{)}
    \PY{c+c1}{\PYZsh{}Calculate inertia}
    \PY{n}{inertia\PYZus{}errors}\PY{o}{.}\PY{n}{append}\PY{p}{(}\PY{n}{model}\PY{o}{.}\PY{n}{named\PYZus{}steps}\PY{p}{[}\PY{l+s+s2}{\PYZdq{}}\PY{l+s+s2}{kmeans}\PY{l+s+s2}{\PYZdq{}}\PY{p}{]}\PY{o}{.}\PY{n}{inertia\PYZus{}}\PY{p}{)}
    \PY{c+c1}{\PYZsh{}Calculate silhoutte score}
    \PY{n}{silhouette\PYZus{}scores}\PY{o}{.}\PY{n}{append}\PY{p}{(}
        \PY{n}{silhouette\PYZus{}score}\PY{p}{(}\PY{n}{X}\PY{p}{,}\PY{n}{model}\PY{o}{.}\PY{n}{named\PYZus{}steps}\PY{p}{[}\PY{l+s+s2}{\PYZdq{}}\PY{l+s+s2}{kmeans}\PY{l+s+s2}{\PYZdq{}}\PY{p}{]}\PY{o}{.}\PY{n}{labels\PYZus{}}\PY{p}{)}
    \PY{p}{)}

    
\PY{n+nb}{print}\PY{p}{(}\PY{l+s+s2}{\PYZdq{}}\PY{l+s+s2}{Inertia:}\PY{l+s+s2}{\PYZdq{}}\PY{p}{,} \PY{n}{inertia\PYZus{}errors}\PY{p}{[}\PY{p}{:}\PY{l+m+mi}{3}\PY{p}{]}\PY{p}{)}
\PY{n+nb}{print}\PY{p}{(}\PY{p}{)}
\PY{n+nb}{print}\PY{p}{(}\PY{l+s+s2}{\PYZdq{}}\PY{l+s+s2}{Silhouette Scores:}\PY{l+s+s2}{\PYZdq{}}\PY{p}{,} \PY{n}{silhouette\PYZus{}scores}\PY{p}{[}\PY{p}{:}\PY{l+m+mi}{3}\PY{p}{]}\PY{p}{)}
\end{Verbatim}
\end{tcolorbox}

    \begin{Verbatim}[commandchars=\\\{\}]
Inertia: [11028.058082607142, 7190.526303575357, 5921.636429837139]

Silhouette Scores: [0.7464502937083215, 0.7044601307791996, 0.6920745639331739]
    \end{Verbatim}

    Let's use plotly express to create a line plot that shows the values of
inertia\_errors as a function of n\_clusters. This is the elbow method
that will help us to determine the number of clusters to use.

    \begin{tcolorbox}[breakable, size=fbox, boxrule=1pt, pad at break*=1mm,colback=cellbackground, colframe=cellborder]
\prompt{In}{incolor}{14}{\boxspacing}
\begin{Verbatim}[commandchars=\\\{\}]
\PY{c+c1}{\PYZsh{} Create line plot of `inertia\PYZus{}errors` vs `n\PYZus{}clusters`}
\PY{n}{fig} \PY{o}{=} \PY{n}{px}\PY{o}{.}\PY{n}{line}\PY{p}{(}
    \PY{n}{x}\PY{o}{=}\PY{n}{n\PYZus{}clusters}\PY{p}{,} \PY{n}{y}\PY{o}{=}\PY{n}{inertia\PYZus{}errors}\PY{p}{,} \PY{n}{title}\PY{o}{=}\PY{l+s+s2}{\PYZdq{}}\PY{l+s+s2}{K\PYZhy{}Means Model: Inertia vs Number of Clusters}\PY{l+s+s2}{\PYZdq{}}
\PY{p}{)}
\PY{n}{fig}\PY{o}{.}\PY{n}{update\PYZus{}layout}\PY{p}{(}\PY{n}{xaxis\PYZus{}title}\PY{o}{=}\PY{l+s+s2}{\PYZdq{}}\PY{l+s+s2}{Number of Clusters (k)}\PY{l+s+s2}{\PYZdq{}}\PY{p}{,} \PY{n}{yaxis\PYZus{}title}\PY{o}{=}\PY{l+s+s2}{\PYZdq{}}\PY{l+s+s2}{Inertia}\PY{l+s+s2}{\PYZdq{}}\PY{p}{)}
\PY{n}{fig}\PY{o}{.}\PY{n}{show}\PY{p}{(}\PY{p}{)}
\end{Verbatim}
\end{tcolorbox}

    
    
    You can see that the line starts to flatten out around 4 or 5 clusters.
Let's make another line plot based on the silhouette scores.

    Let's use plotly express to create a line plot that shows the values of
silhouette\_scores as a function of n\_clusters

    \begin{tcolorbox}[breakable, size=fbox, boxrule=1pt, pad at break*=1mm,colback=cellbackground, colframe=cellborder]
\prompt{In}{incolor}{15}{\boxspacing}
\begin{Verbatim}[commandchars=\\\{\}]
\PY{c+c1}{\PYZsh{} Create a line plot of `silhouette\PYZus{}scores` vs `n\PYZus{}clusters`}
\PY{n}{fig} \PY{o}{=} \PY{n}{px}\PY{o}{.}\PY{n}{line}\PY{p}{(}
    \PY{n}{x}\PY{o}{=}\PY{n}{n\PYZus{}clusters}\PY{p}{,}
    \PY{n}{y}\PY{o}{=}\PY{n}{silhouette\PYZus{}scores}\PY{p}{,}
    \PY{n}{title}\PY{o}{=}\PY{l+s+s2}{\PYZdq{}}\PY{l+s+s2}{K\PYZhy{}Means Model: Silhouette Score vs Number of Clusters}\PY{l+s+s2}{\PYZdq{}}
\PY{p}{)}
\PY{n}{fig}\PY{o}{.}\PY{n}{update\PYZus{}layout}\PY{p}{(}
    \PY{n}{xaxis\PYZus{}title}\PY{o}{=}\PY{l+s+s2}{\PYZdq{}}\PY{l+s+s2}{Number of Clusters}\PY{l+s+s2}{\PYZdq{}}\PY{p}{,} \PY{n}{yaxis\PYZus{}title}\PY{o}{=}\PY{l+s+s2}{\PYZdq{}}\PY{l+s+s2}{Silhouette Score}\PY{l+s+s2}{\PYZdq{}}
\PY{p}{)}

\PY{n}{fig}\PY{o}{.}\PY{n}{show}\PY{p}{(}\PY{p}{)}
\end{Verbatim}
\end{tcolorbox}

    
    
    We can see that the best silhouette scores occur when there are 3 or 4
clusters.

Putting the information from this plot together with our inertia plot,
it seems like the best setting for \texttt{n\_clusters} will be 4.

    Let's build and train a new k-means model named final\_model,using the
information we gained from the two plots above to set an appropriate
value for the n\_clusters argument.

    \begin{tcolorbox}[breakable, size=fbox, boxrule=1pt, pad at break*=1mm,colback=cellbackground, colframe=cellborder]
\prompt{In}{incolor}{16}{\boxspacing}
\begin{Verbatim}[commandchars=\\\{\}]
\PY{n}{final\PYZus{}model} \PY{o}{=} \PY{n}{make\PYZus{}pipeline}\PY{p}{(}
    \PY{n}{StandardScaler}\PY{p}{(}\PY{p}{)}\PY{p}{,}
    \PY{n}{KMeans}\PY{p}{(}\PY{n}{n\PYZus{}clusters}\PY{o}{=}\PY{l+m+mi}{4}\PY{p}{,} \PY{n}{random\PYZus{}state}\PY{o}{=}\PY{l+m+mi}{42}\PY{p}{)}
\PY{p}{)}
\PY{n}{final\PYZus{}model}\PY{o}{.}\PY{n}{fit}\PY{p}{(}\PY{n}{X}\PY{p}{)}
\end{Verbatim}
\end{tcolorbox}

            \begin{tcolorbox}[breakable, size=fbox, boxrule=.5pt, pad at break*=1mm, opacityfill=0]
\prompt{Out}{outcolor}{16}{\boxspacing}
\begin{Verbatim}[commandchars=\\\{\}]
Pipeline(steps=[('standardscaler', StandardScaler()),
                ('kmeans', KMeans(n\_clusters=4, random\_state=42))])
\end{Verbatim}
\end{tcolorbox}
        
    \hypertarget{communicate}{%
\section{Communicate}\label{communicate}}

    \begin{tcolorbox}[breakable, size=fbox, boxrule=1pt, pad at break*=1mm,colback=cellbackground, colframe=cellborder]
\prompt{In}{incolor}{17}{\boxspacing}
\begin{Verbatim}[commandchars=\\\{\}]
\PY{c+c1}{\PYZsh{}Extraxting labels created by final\PYZus{}model during training}
\PY{n}{labels} \PY{o}{=} \PY{n}{final\PYZus{}model}\PY{o}{.}\PY{n}{named\PYZus{}steps}\PY{p}{[}\PY{l+s+s2}{\PYZdq{}}\PY{l+s+s2}{kmeans}\PY{l+s+s2}{\PYZdq{}}\PY{p}{]}\PY{o}{.}\PY{n}{labels\PYZus{}}
\PY{n+nb}{print}\PY{p}{(}\PY{n}{labels}\PY{p}{[}\PY{p}{:}\PY{l+m+mi}{5}\PY{p}{]}\PY{p}{)}
\end{Verbatim}
\end{tcolorbox}

    \begin{Verbatim}[commandchars=\\\{\}]
[0 0 0 0 0]
    \end{Verbatim}

    \begin{tcolorbox}[breakable, size=fbox, boxrule=1pt, pad at break*=1mm,colback=cellbackground, colframe=cellborder]
\prompt{In}{incolor}{18}{\boxspacing}
\begin{Verbatim}[commandchars=\\\{\}]
\PY{c+c1}{\PYZsh{} DataFrame xgb that contains the mean values of the features in X for each of the clusters in final\PYZus{}model}
\PY{n}{xgb} \PY{o}{=} \PY{n}{X}\PY{o}{.}\PY{n}{groupby}\PY{p}{(}\PY{n}{labels}\PY{p}{)}\PY{o}{.}\PY{n}{mean}\PY{p}{(}\PY{p}{)}
\PY{n}{xgb}
\end{Verbatim}
\end{tcolorbox}

            \begin{tcolorbox}[breakable, size=fbox, boxrule=.5pt, pad at break*=1mm, opacityfill=0]
\prompt{Out}{outcolor}{18}{\boxspacing}
\begin{Verbatim}[commandchars=\\\{\}]
            DEBT       NETWORTH         HOUSES          NFIN         ASSET
0   25475.184486   13336.327716   12657.640906  2.604130e+04  3.881151e+04
1  126752.280000  900314.904000  289648.000000  7.349837e+05  1.027067e+06
2  215657.219388  161855.510204  248186.862245  3.194630e+05  3.775127e+05
3  725213.134328  778260.298507  819776.119403  1.289561e+06  1.503473e+06
\end{Verbatim}
\end{tcolorbox}
        
    Let's use plotly express to create a side-by-side bar chart from xgb
that shows the mean of the features in X for each of the clusters in
final\_model

    \begin{tcolorbox}[breakable, size=fbox, boxrule=1pt, pad at break*=1mm,colback=cellbackground, colframe=cellborder]
\prompt{In}{incolor}{19}{\boxspacing}
\begin{Verbatim}[commandchars=\\\{\}]
\PY{c+c1}{\PYZsh{} Create side\PYZhy{}by\PYZhy{}side bar chart of `xgb`}
\PY{n}{fig} \PY{o}{=} \PY{n}{px}\PY{o}{.}\PY{n}{bar}\PY{p}{(}
    \PY{n}{xgb}\PY{p}{,}
    \PY{n}{barmode}\PY{o}{=}\PY{l+s+s2}{\PYZdq{}}\PY{l+s+s2}{group}\PY{l+s+s2}{\PYZdq{}}\PY{p}{,}
    \PY{n}{title}\PY{o}{=}\PY{l+s+s2}{\PYZdq{}}\PY{l+s+s2}{Mean Household Finances by Cluster}\PY{l+s+s2}{\PYZdq{}}
\PY{p}{)}
\PY{n}{fig}\PY{o}{.}\PY{n}{update\PYZus{}layout}\PY{p}{(}\PY{n}{xaxis\PYZus{}title}\PY{o}{=}\PY{l+s+s2}{\PYZdq{}}\PY{l+s+s2}{Cluster}\PY{l+s+s2}{\PYZdq{}}\PY{p}{,} \PY{n}{yaxis\PYZus{}title}\PY{o}{=}\PY{l+s+s2}{\PYZdq{}}\PY{l+s+s2}{Value [\PYZdl{}]}\PY{l+s+s2}{\PYZdq{}}\PY{p}{)}
\PY{n}{fig}\PY{o}{.}\PY{n}{show}\PY{p}{(}\PY{p}{)}
\end{Verbatim}
\end{tcolorbox}

    
    
    Remember that our clusters are based partially on \texttt{NETWORTH},
which means that the households in the 0 cluster have the smallest net
worth, and the households in the 2 cluster have the highest. Based on
that, there are some interesting things to unpack here.

First, let's look at the \texttt{DEBT} variable. One might think that it
would scale as net worth increases, but it doesn't. The lowest amount of
debt is carried by the households in cluster 2, even though the value of
their houses (shown in green) is roughly the same. You can't
\emph{really} tell from this data what's going on, but one possibility
might be that the people in cluster 2 have enough money to pay down
their debts, but not quite enough money to leverage what they have into
additional debts. The people in cluster 3, by contrast, might not need
to worry about carrying debt because their net worth is so high.

Finally, since we started out this project looking at home values, take
a look at the relationship between \texttt{DEBT} and \texttt{HOUSES}.
The value of the debt for the people in cluster 0 is higher than the
value of their houses, suggesting that most of the debt being carried by
those people is tied up in their mortgages --- if they own a home at
all. Contrast that with the other three clusters: the value of everyone
else's debt is lower than the value of their homes.

    Let's create a PCA transformer, use it to reduce the dimensionality of
the data in X to 2, and then put the transformed data into a DataFrame
named X\_pca. The columns of X\_pca should be named ``PC1'' and ``PC2''

    \begin{tcolorbox}[breakable, size=fbox, boxrule=1pt, pad at break*=1mm,colback=cellbackground, colframe=cellborder]
\prompt{In}{incolor}{20}{\boxspacing}
\begin{Verbatim}[commandchars=\\\{\}]
\PY{c+c1}{\PYZsh{} Instantiate transformer}
\PY{n}{pca} \PY{o}{=} \PY{n}{PCA}\PY{p}{(}\PY{n}{n\PYZus{}components}\PY{o}{=}\PY{l+m+mi}{2}\PY{p}{,} \PY{n}{random\PYZus{}state}\PY{o}{=}\PY{l+m+mi}{42}\PY{p}{)}

\PY{c+c1}{\PYZsh{} Transform `X`}
\PY{n}{X\PYZus{}t} \PY{o}{=} \PY{n}{pca}\PY{o}{.}\PY{n}{fit\PYZus{}transform}\PY{p}{(}\PY{n}{X}\PY{p}{)}

\PY{c+c1}{\PYZsh{} Put `X\PYZus{}t` into DataFrame}
\PY{n}{X\PYZus{}pca} \PY{o}{=} \PY{n}{pd}\PY{o}{.}\PY{n}{DataFrame}\PY{p}{(}\PY{n}{X\PYZus{}t}\PY{p}{,} \PY{n}{columns}\PY{o}{=}\PY{p}{[}\PY{l+s+s2}{\PYZdq{}}\PY{l+s+s2}{PC1}\PY{l+s+s2}{\PYZdq{}}\PY{p}{,}\PY{l+s+s2}{\PYZdq{}}\PY{l+s+s2}{PC2}\PY{l+s+s2}{\PYZdq{}}\PY{p}{]}\PY{p}{)}

\PY{n+nb}{print}\PY{p}{(}\PY{l+s+s2}{\PYZdq{}}\PY{l+s+s2}{X\PYZus{}pca shape:}\PY{l+s+s2}{\PYZdq{}}\PY{p}{,} \PY{n}{X\PYZus{}pca}\PY{o}{.}\PY{n}{shape}\PY{p}{)}
\PY{n}{X\PYZus{}pca}\PY{o}{.}\PY{n}{head}\PY{p}{(}\PY{p}{)}
\end{Verbatim}
\end{tcolorbox}

    \begin{Verbatim}[commandchars=\\\{\}]
X\_pca shape: (4418, 2)
    \end{Verbatim}

            \begin{tcolorbox}[breakable, size=fbox, boxrule=.5pt, pad at break*=1mm, opacityfill=0]
\prompt{Out}{outcolor}{20}{\boxspacing}
\begin{Verbatim}[commandchars=\\\{\}]
             PC1           PC2
0 -221525.424530 -22052.273003
1 -217775.100722 -22851.358068
2 -219519.642175 -19023.646333
3 -212195.720367 -22957.107039
4 -215540.507551 -20259.749306
\end{Verbatim}
\end{tcolorbox}
        
    Let's use plotly express to create a scatter plot of X\_pca using
seaborn. Be sure to color the data points using the labels generated by
your final\_model

    \begin{tcolorbox}[breakable, size=fbox, boxrule=1pt, pad at break*=1mm,colback=cellbackground, colframe=cellborder]
\prompt{In}{incolor}{21}{\boxspacing}
\begin{Verbatim}[commandchars=\\\{\}]
\PY{c+c1}{\PYZsh{} Create scatter plot of `PC2` vs `PC1`}
\PY{n}{fig} \PY{o}{=} \PY{n}{px}\PY{o}{.}\PY{n}{scatter}\PY{p}{(}
    \PY{n}{data\PYZus{}frame}\PY{o}{=}\PY{n}{X\PYZus{}pca}\PY{p}{,}
    \PY{n}{x}\PY{o}{=}\PY{l+s+s2}{\PYZdq{}}\PY{l+s+s2}{PC1}\PY{l+s+s2}{\PYZdq{}}\PY{p}{,}
    \PY{n}{y}\PY{o}{=}\PY{l+s+s2}{\PYZdq{}}\PY{l+s+s2}{PC2}\PY{l+s+s2}{\PYZdq{}}\PY{p}{,}
    \PY{n}{color}\PY{o}{=}\PY{n}{labels}\PY{o}{.}\PY{n}{astype}\PY{p}{(}\PY{n+nb}{str}\PY{p}{)}\PY{p}{,}
    \PY{n}{title}\PY{o}{=}\PY{l+s+s2}{\PYZdq{}}\PY{l+s+s2}{PCA Representation of Clusters}\PY{l+s+s2}{\PYZdq{}}
\PY{p}{)}
\PY{n}{fig}\PY{o}{.}\PY{n}{update\PYZus{}layout}\PY{p}{(}\PY{n}{xaxis\PYZus{}title}\PY{o}{=}\PY{l+s+s2}{\PYZdq{}}\PY{l+s+s2}{PC1}\PY{l+s+s2}{\PYZdq{}}\PY{p}{,} \PY{n}{yaxis\PYZus{}title}\PY{o}{=}\PY{l+s+s2}{\PYZdq{}}\PY{l+s+s2}{PC2}\PY{l+s+s2}{\PYZdq{}}\PY{p}{)}
\PY{n}{fig}\PY{o}{.}\PY{n}{show}\PY{p}{(}\PY{p}{)}
\end{Verbatim}
\end{tcolorbox}

    
    
    \begin{tcolorbox}[breakable, size=fbox, boxrule=1pt, pad at break*=1mm,colback=cellbackground, colframe=cellborder]
\prompt{In}{incolor}{ }{\boxspacing}
\begin{Verbatim}[commandchars=\\\{\}]

\end{Verbatim}
\end{tcolorbox}


    % Add a bibliography block to the postdoc
    
    
    
\end{document}
